
\section{Question 1}
MAC destination: 90 06 28 5E 4C 01 
MAC source: 64 5A 04 BA F2 31
Protocol de couche supérieur: 08 00 (IPv4)
\section{Question 2}
L'adresse IP permet de déterminer où exactement envoyer l'information  en donnant des détails sur la localisation géographique, sur comment l'ordinateur est connecté a un réseau et d'identifier l'ordinateur sur le réseau.

La seule façon de faire fonctionner un réseau en
utilisant uniquement des adresses MAC serait de
stocker toutes les adresses MAC existantes dans
chaque routeurs, ce qui est pas pratique.

\section{Question 3}
\begin{itemize}
	\item Lorsqu'un ordinateur veut envoyer un paquet à un autre en connaissant uniquement son adresse IP (et non son adresse MAC), il envoie une requête avec FF:FF:FF:FF:FF:FF comme adresse MAC de destination. Celle-ci est reconnue comme une requête d'adresse MAC par tous les ordinateurs et ceux-ci renvoient leur adresse MAC ainsi que leur adresse IP.
	\item adresse MAC : 90:06:28:5e:4c:01
	adresse IP : 10.248.100.164
	\item Non car après avoir reçu la réponse de
	la requète, l'adresse MAC est stockée en cache
	 avec l'adresse IP correspondante.
\end{itemize}

\section{Question 4}
\begin{itemize}
	\item 
\end{itemize}
