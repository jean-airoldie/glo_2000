
\section{Question 1}
\begin{enumerate}[(a)]
	\item 
		\[ P(\%) = f(m, p, h_1, h_2, h_3) = 
		\frac{m}{(\frac{8*2p}{3}h_1 + \frac{8*p}{6}h_2 + \frac{8*p}{6}h_3 + m)} 100\%\]
	
	\item 
		En assumant que le meme nombre de couches qui ajoutent $h_3$ reste le même qu'à la
		question a) :
		\[ P(\%) = f(m, p, h_1, h_2, h_3, h_4) = 
		\frac{m}{(\frac{8*2p}{3}h_1 + \frac{8*p}{6}h_2 + \frac{8*p}{6}h_3 + \frac{8*p}{6}h_4 + m)} 100\%\]
	En assumant que le nombre de couches qui ajoutent $h_3$ est le nombre de couches restantes :  
		\[ P(\%) = f(m, p, h_1, h_2, h_3, h_4) = 
		\frac{m}{(\frac{8*2p}{3}h_1 + \frac{8*p}{6}h_2 + 0h_3 + \frac{8*p}{6}h_4 + m)} 100\%\]
	\item
		\[ P(\%) = f(16000, 6, 128, 256, 0, 64) = 
		\frac{16000}{(\frac{8*2*6}{3}128 + \frac{8*6}{6}256 + \frac{8*6}{6}64 + 16000)} 100\% = 70.62\%\]
\end{enumerate}

\section{Question 2A}

\begin{enumerate}[(a)]
	\item
		\parbox{\linewidth}{\centering
		\includegraphics[page=1,scale=0.7]{julesScan}
		}
	\item
		\parbox{\linewidth}{\centering
		\includegraphics[page=2,scale=0.7]{julesScan}
		}

\end{enumerate}

\pagebreak

\section{Question 2B}

\begin{enumerate}[(a)]
	\item 
		\[R_m = 1500\]
		\[D = R_m\log_2 V = 1500\log_2 4 = 3000bits/sec\]
		Ou $V$ est égale au nombre de valeurs
	\item
		\parbox{\linewidth}{\centering
		\includegraphics[page=4,scale=0.7]{julesScan}
		}
		
\end{enumerate}

\section{Question 3}

\begin{enumerate}[(a)]
	\item
		\parbox{\linewidth}{\centering
		\includegraphics[page=5,scale=0.7]{julesScan}
		}
	\item
		\parbox{\linewidth}{\centering
		\includegraphics[page=6,scale=0.7]{julesScan}
		}
\end{enumerate}

%stuff

\section{Question 4}
\begin{enumerate}[(a)]
	\item  $T_{x} = \frac{Taille_{trame}}{vitesse_{transmission}}$ \\
Donc, l'utilisation max du canal = $\frac{T_{x}}{T_{x}+T_{propagation}}$ 
		=  $\frac{1 ms}{1+(2 * 250ms)} \approx 0,2\%$
	\item Taux d'utilisation maximal = $\frac{\omega}{1+2BD}$, où 
		$\omega = 2^n-1 = 2^(3bits)-1 = 7$\\
		Donc, l'utilisation max du canal = $\frac{7}{1+(2 * 250ms)} \approx 1,397\%$
	\item C'est le même calcul, mais le calcul d'oméga change:
		$\omega = 2^{(n-1)} = 2^{(3-1bits)} = 4  trames$\\
		Donc, l'utilisation max du canal = $\frac{4}{1+(2 * 250ms)} \approx 0,798\%$
\end{enumerate}

\section{Question 5}
\parbox{\linewidth}{\centering
\includegraphics[scale=0.5]{question5}
}

