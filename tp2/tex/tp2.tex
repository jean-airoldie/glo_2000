
\section{Question 1}
\begin{enumerate}
	\item 
	\begin{enumerate}[a)]
		\item 
			commande : \verb!sudo nping www.cisco.com --tcp -g 1337 -p 80 --flags SYN -c 1!\\
			nom du fichier trace réseau : TP2\_Q1\_a.pcapng
		\item 
			commande : \verb!sudo nping wikipedia.org --udp -g 9999 -p 80 -c 1!\\
			nom du fichier trace réseau : TP2\_Q1\_b.pcapng
		\item 
			commande : \verb!sudo nping www.imdb.com --tcp -g 11111 -p 80 --flag RST -c 1!\\
			nom du fichier trace réseau : TP2\_Q1\_c.pcapng
	\end{enumerate}
	\item 
		\begin{enumerate}
			\item
				Le port de destination (-p) doit être 80 
				car c'est le port que les serveurs ecoutent.
			\item
				L'adresse ip source (-S) n'est pas la bonne,
				la réponse ne sera donc pas envoyer à la bonne
				adresse.
			\item
				Le time to live du packet (--ttl) est trop court
				pour se rendre a destination.
		\end{enumerate}
\end{enumerate}

\section{Question 2}
C'est un programme de type traceroute qui envoit des packets, 
en incrementant le TTL jusqu'a ce que l'adresse de destination soit 
atteinte, afin de connaitre le chemin emprunter par les packets. 
A chaque fois qu'un routeur recoit un packet, 
celui-ci decrement le TTL, et lorsque celui-ci atteint 0,
le routeur envoit un packet ICMP vers l'adresse source pour lui
communiquer l'erreur "Time-to-live-exceeded". Ainsi, lorsque 
l'adresse source recoit ces packets ICMP, 
elle peux connaitre le chemin du packet, étape par étape.

\section{Question 3}

\section{Question 4}
\begin{enumerate}
	\item Le mode TCP génère deux paquets alors que le mode UDP en génère qu'un seul. C'est parce que le recepteur du paquet en mode TCP envoie un ACK pour signifier à l'émetteur qu'il a bien reçu le paquet. À l'inverse, l'émetteur en UDP ne sait pas si le paquet s'est rendu à destination. Il n'y a aucune correction d'erreur.
	\item La commande pour démarrer le serveur dans Ncat est : ncat -u -l -p 1337 \\ La commande pour envoyer 5 fois le mot bonjour est : \\ nping --udp -p 1337 -c 5 --data-string "bonjour" localhost
\end{enumerate}

