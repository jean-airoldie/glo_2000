
\section{Question 1}
\begin{enumerate}
	\item 
	\begin{enumerate}[a)]
		\item 
			commande : \verb!sudo nping www.cisco.com --tcp -g 1337 -p 80 --flags SYN -c 1!\\
			nom du fichier trace réseau : TP2\_Q1\_a.pcapng
		\item 
			commande : \verb!sudo nping wikipedia.org --udp -g 9999 -p 80 -c 1!\\
			nom du fichier trace réseau : TP2\_Q1\_b.pcapng
		\item 
			commande : \verb!sudo nping www.imdb.com --tcp -g 11111 -p 80 --flag RST -c 1!\\
			nom du fichier trace réseau : TP2\_Q1\_c.pcapng
	\end{enumerate}
	\item 
		\begin{enumerate}
			\item
				Le port de destination (-p) doit être 80 
				car c'est le port que les serveurs ecoutent.
			\item
				L'adresse ip source (-S) n'est pas la bonne,
				la réponse ne sera donc pas envoyer à la bonne
				adresse.
			\item
				Le time to live du packet (--ttl) est trop court
				pour se rendre a destination.
		\end{enumerate}
\end{enumerate}

\section{Question 2}
C'est un programme de type traceroute qui envoit plusieurs packets, 
en incrementant le TTL afin de connaitre le chemin emprunter 
par les packets vers une adresse de destination. 
A chaque fois qu'un routeur recoit un packet, 
celui-ci decrement le TTL, et lorsque celui-ci atteint 0,
le routeur envoit un packet ICMP vers l'adresse source pour lui
communiquer l'erreur "Time-to-live-exceeded". Ainsi lorsque l'adresse
source recoit ces packets ICMP, elle peux connaitre le chemin du
packet, étape par étape.

\section{Question 3}

\section{Question 4}
\begin{enumerate}
	\item 
	\item 
\end{enumerate}

