
\section{Question 1}
MAC destination: 90 06 28 5E 4C 01\\ 
MAC source: 64 5A 04 BA F2 31\\
Protocole de couche supérieure: 08 00 (IPv4)

\section{Question 2}
L'adresse IP permet de déterminer où exactement envoyer l'information  en donnant des détails sur la localisation géographique, sur comment l'ordinateur est connecté a un réseau et d'identifier l'ordinateur sur le réseau.

La seule façon de faire fonctionner un réseau en
utilisant uniquement des adresses MAC serait de
stocker toutes les adresses MAC existantes dans le monde
dans chaque routeur, ce qui n’est évidamment pas pratique.

\section{Question 3}
\begin{enumerate}[(a)]
	\item Lorsqu'un ordinateur veut envoyer un paquet à un autre en connaissant uniquement son adresse IP (et non son adresse MAC), il envoie une requête avec FF:FF:FF:FF:FF:FF comme adresse MAC de destination. Celle-ci est reconnue comme une requête d'adresse MAC par tous les ordinateurs et ceux-ci renvoient leur adresse MAC ainsi que leur adresse IP.
	\item adresse MAC : 90:06:28:5e:4c:01
	adresse IP : 10.248.100.164
	\item Non, car après avoir reçu la réponse de
	la requête, l'adresse MAC est stockée en cache
	 avec l'adresse IP correspondante.
\end{enumerate}

\section{Question 4}
\begin{enumerate}[(a)]
	\item Paquet 1: Time to live décimal : 64. En format hexadécimal, cela fait 40. \\
	Paquet 2: Time to live décimal : 240. En format hexadécimal, cela fait f0.
	\item La valeur du Time to live peut être entre 0 et 255. Cela fait 256, ou 2⁸ valeurs possibles. La valeur tient donc sur 8 bits.

	\item Le deuxième paquet a traversé 15 routeurs avant d'atteindre 10.0.9.128.
\end{enumerate}
\section{Question 5}
L'adresse 192.168.1.148 est l'ordinateur qui effectue le ping et qui envoie la requête. Celle-ci passe par le routeur à l'adresse 192.168.1.1 qui ensuite retransmet le paquet à l'adresse de destination, 172.217.1.68, qui correspond à Google. Le destinataire répond alors à l'ordinateur à l'adresse 192.168.1.148, ce qui permet à l'ordinateur source de déterminer le temps de réponse.
\section{Question 6}
Les réponses sont de type 3 (Destination unreachable) et le code est 13 (Communication administratively filtered).
Une cause possible serait un pare-feu administratif qui bloque la requête.
